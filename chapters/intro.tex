\chapter{Introduction}
\label{chap:introduction}

The University of West Florida autonomous golf cart project is intended to be
a platform for autonomous navigation. The vehicle is a
utility class electric golf cart controlled with drive-by-wire actuators and an
embedded compute unit. This work focuses on the control and software layers
that connect those actuators to high level autonomous navigation algorithms.

A TI F28379D C2000 microcontroller executes low level control for traction, braking, and steering.
A NVIDIA Jetson Orin Nano runs ROS~2 and hosts higher
level control algorithms. A Gazebo-based simulation mirrors the physical system and
utilizes the same command and feedback interfaces, so control software can be
developed and exercised in simulation before deployment to the vehicle.

\section{Motivation and Objectives}
\label{sec:intro_motivation}

Autonomous navigation experiments require a control system that behaves in a
predictable way and can be reused across different algorithms. Separating low
level actuation from high level decision making helps with both. The
microcontroller is responsible for tracking speed and steering commands. The
Jetson is responsible for deciding which commands to send.

The objectives of this project are to
\begin{itemize}
    \item derive longitudinal and planar vehicle models suitable for control
          design and simulation,
    \item design and implement low level controllers for drive, brake, and
          steering on the C2000,
    \item develop a Gazebo-based simulation that uses the same command
          interface as the physical cart,
    \item implement ROS~2 nodes for key sensors and for the C2000 link,
    \item define and begin implementing a high level controller for GPS-based
          autonomous navigation.
\end{itemize}

These pieces form the control and software infrastructure needed for later
work on path planning, perception, and higher level autonomy.

\section{Scope}
\label{sec:intro_scope}

The scope of this report is limited to the control and software that bridge
the existing mechanical platform and autonomous navigation. The work includes
\begin{itemize}
    \item vehicle modeling for longitudinal dynamics and planar motion,
    \item low level PI control design and embedded implementation on the
          C2000,
    \item construction of a Gazebo simulation with a URDF model, Ackermann
          steering plugin, and ROS~2 bridge,
    \item ROS~2 integration on the Jetson for GPS, compass, and the C2000
          serial interface,
    \item architecture and partial implementation of a GPS waypoint follower.
\end{itemize}

This project does not address ride-share scheduling, multi-vehicle
coordination, or operation at higher speeds. Perception and obstacle
avoidance are only treated at the architectural level. GPS-based waypoint
navigation on the physical cart was a target, but only part of the high level
functionality was completed.

\section{Contributions}
\label{sec:intro_contributions}

Within this scope, the contributions of this work to the autonomous golf cart
platform are
\begin{itemize}
    \item a set of longitudinal and planar models tied to measured vehicle
          parameters and used directly for controller design and simulation,
    \item a low level controller implementation on the C2000 for traction,
          braking, and steering, with a unified serial command interface,
    \item a Gazebo simulation of the cart that shares packet formats and ROS~2
          topics with the physical system,
    \item a ROS~2-based high level control structure on the Jetson that
          supports GPS-based autonomous navigation and future perception
          modules.
\end{itemize}
