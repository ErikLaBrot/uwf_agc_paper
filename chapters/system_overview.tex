\chapter{System Overview}
[High level description for the golf cart]
The platform consists of the combination of the physical golf cart, and the modifications made
to the original design. Since its acquisition from the manufacturer, the platform has undergone
several alterations and overhauls that occured prior to this project. A summarization of the current state 
of the platform is the focus of this chapter.

\section{Vehicle Platform}
[Physical platform as received, brief overview of modifications physical modifications]
The physical component of the platform is comprised of a modified Club Car Pioneer 1200. Originally, the platform
was an internal combustion engine powered utility class golf cart. Since then, it has been modified in the following
manners:
- The engine has been swapped out for a 10kW permanent magnet AC motor
- A 48V sinusoidal motor controller with supporting electronics has been installed
- The steering column has been modified with a linkage to a DC motor
- A linear actuation system has been attached to the vehicle's brake actuation pedal

\section{Sensing and Computation}
[Describe sensors and computational hardware on the vehicle]
At the beginning of this stage of the project, the golf cart computational stack was comprised of 
multiple 8-bit microcontrollers to interface with actuators slaved to a 32-bit microcontroller. These were
the original low level control system's computer hardware. This has since been replaced with a single 32-bit 
microcontroller, a TI F28379D launchpad based on the C2000 architecture.

Additionally, the platform had an Apple iMac interfaced with a raspberry pi to comprise the high-level control systemm.
This has been replaced with a NVIDIA Jetson Orin Nano as the high level controller.

For sensors, the platform has the following:
- Sick LMS200 LiDAR
- GPS

\section{Control Architecture}
[Describe the control architecture in terms of components (e.g.\ high-level, low level, simulation, etc)]
A brief overview of the control systems can be viewed as high-level, low-level, and platform level. 
The high-level control system focuses on controlling position of the cart, referenced externally to GPS, and performing
computationally taxing tasks such as obstacle avoidance and path planning.

The low-level control system focuses on ensuring that the speed and steering angle control of the platform is sufficiently 
linearized such that high level control is simplified. This acts as a translation layer that will allow the high level control 
system to command a velocity setpoint for the platform, and the appropriate forces will be applied to the wheels and steering 
columns such that the velocity is met. 

The platform level is comprised of the motors, drivetrains, and motor controllers attached to the cart. These represent the 
kinematics and dynamics of the golf cart, and encompass how when a specific torque is commanded from a motor, how that torque is 
produced.