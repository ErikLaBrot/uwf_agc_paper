\chapter{System Overview}
\label{chap:system_overview}

The autonomous golf cart consists of the physical vehicle, the onboard sensing
and compute hardware, and the control software that connects them. The original
gasoline Club Car Pioneer 1200 has been rebuilt as an electric, drive-by-wire
platform. This chapter summarizes the current system without going into detailed
modeling or controller design.

\section{Vehicle Platform}
The base vehicle is a Club Car Pioneer 1200 utility golf cart. In its original
form it used an internal combustion engine and conventional driver-operated
steering and braking. For this project it has been converted to an electric
platform with electronically controlled traction, steering, and braking.

The major physical and drivetrain modifications are:
%
\begin{itemize}
    \item The internal combustion engine has been removed and replaced with a
          \SI{10}{\kilo\watt} permanent magnet AC traction motor.
    \item A \SI{48}{\volt} sinusoidal motor controller and supporting power
          electronics have been installed to drive the traction motor.
    \item The steering column has been fitted with a mechanical linkage to a
          DC motor so that steering angle can be commanded electronically.
    \item A linear actuator has been attached to the brake pedal to apply and
          release the friction brake under computer control.
\end{itemize}

These changes allow drive torque, steering angle, and brake force to be commanded
by a control system rather than directly by a human driver. The rest of the
vehicle (chassis, suspension, and basic mechanical layout) remains close to the
original Pioneer 1200 design.

\section{Sensing and Computation}
The platform includes a set of sensors and computation hardware to support
autonomous operation.

On the computation side, the current system uses:
%
\begin{itemize}
    \item A TI F28379D LaunchPad based on the C2000 architecture as the low level
          controller. It interfaces with the traction motor controller, brake
          actuator, throttle pedal, steering motor, and steering encoder, and
          handles real-time control tasks.
    \item An NVIDIA Jetson Orin Nano as the high level controller. It runs the
          navigation and supervisory software and communicates with the low level
          controller over a defined interface.
\end{itemize}

On the sensing side, the platform currently uses:
%
\begin{itemize}
    \item A SICK LMS200 LiDAR for planar range measurements and obstacle
          detection in front of the vehicle.
    \item A GPS receiver for global position estimates used in waypoint
          navigation.
\end{itemize}

Additional internal signals such as motor controller status and encoder feedback
are also available to the control system through the low level controller. These
are used for state estimation, monitoring, and closed loop control.

\section{Control Architecture}
From a control point of view, the system is organized into three layers:
a high level control layer, a low level control layer, and the physical platform.

The \emph{high level} control layer runs on the Jetson Orin Nano. Its main job is
to control the position of the cart in the environment. It receives GPS and LiDAR
data, accepts a sequence of waypoints, plans a path, and generates desired speed
and steering commands. This layer can also host more computationally expensive
algorithms such as obstacle avoidance and future mapping or localization methods.

The \emph{low level} control layer runs on the TI F28379D C2000 microcontroller.
It takes the desired speed and steering commands from the high level controller
and converts them into actuator signals. In practice, it closes feedback loops on
vehicle speed and steering angle so that, from the perspective of the high level
control, the vehicle behaves like a system with simple setpoints rather than raw
motor voltages or currents.

The \emph{platform} layer is the physical hardware: the traction motor and its
controller, the steering motor and linkage, the brake actuator, the drivetrain,
and the vehicle chassis. This layer determines how commanded torques and forces
translate into motion. The interaction between the low level controller and this
hardware is the subject of the modeling and controller design discussed in later
chapters.

Communication between layers is through well-defined signals. The high level
controller sends speed and steering setpoints to the low level controller and
receives status and diagnostic information. The low level controller commands the
platform actuators and monitors their feedback. This layered structure is intended
to keep the system modular and make it easier to extend and maintain.
