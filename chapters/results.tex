\chapter{Results}
\label{chap:results}

This chapter summarizes the validation of the models, controllers, and
software architecture developed in the previous chapters. Results are
organized into two main parts:

\begin{itemize}
    \item Simulation-based validation of the low-level controllers inside the
          Gazebo environment described in Chapter~\ref{chap:simulation},
    \item Hardware-in-the-loop and bench tests of the C2000 controller and
          drive-by-wire hardware.
\end{itemize}

The intent is to provide a concise summary of controller behavior and system
integration. Step response plots, time histories, and block diagrams are
intended to be placed throughout this chapter to illustrate the results.

% Figure ideas:
% - Simulated speed step response (drive controller)
% - Simulated brake response (commanded deceleration / speed-to-zero)
% - Simulated steering angle step response
% - HIL test setup photo of cart and instrumentation
% - Measured speed step response on hardware
% - Measured steering step response on hardware

\section{Simulation-Based Controller Validation}
\label{sec:sim_validation}

Simulation results were generated using the Gazebo-based HIL environment from
Chapter~\ref{chap:simulation}. In that setup, the same Simulink model that
runs on the C2000 microcontroller is driven through the serial bridge and
controls the Gazebo vehicle model via ROS~2. This makes the simulated
responses representative of the actual firmware and command interface.

All controller tests in this section use the planar model from
Chapter~\ref{modeling} and the low-level controller structure from
Chapter~\ref{chap:low_level_control_design}. The focus is on step responses
and qualitative behavior.

\subsection{Longitudinal Speed Controller}
\label{sec:sim_speed_results}

The drive speed controller was evaluated in simulation by commanding a step in
desired speed from standstill to a moderate cruising speed. The test procedure
is:

\begin{enumerate}
    \item Initialize the vehicle at rest on level ground,
    \item Apply a step in speed command $v^*$ from $0$ to the target value,
    \item Record vehicle speed, throttle command, and brake command over time.
\end{enumerate}

The throttle command saturates briefly during the initial acceleration, then
backs off as speed approaches the target. The brake command remains at zero
throughout the maneuver, confirming that the domain logic for mutually
exclusive throttle and brake is functioning as intended.

% Figure: Simulated speed step response (velocity vs time, throttle command)
\begin{figure}[h]
    \centering
    % TODO replace with actual Simulink screenshot
    \includegraphics[width=\linewidth]{figures/Results/Figure_1.png}
    \caption{Results for Step Response to Drive Controller}
    \label{fig:drive_response}
\end{figure}
\clearpage

\subsection{Brake Controller}
\label{sec:sim_brake_results}

The brake controller was evaluated by commanding a speed step from a steady
cruise back to zero. The test procedure mirrors the drive test, but with the
cart starting at a fixed speed and a step commanded down to $v^* = 0$:

\begin{enumerate}
    \item Accelerate the vehicle to a steady cruising speed using the drive
          controller,
    \item At $t = 0$, command a step to zero speed,
    \item Record speed, brake command, and throttle command.
\end{enumerate}

In simulation, the brake controller brings the speed to zero in a controlled
manner. The brake command ramps up to a level that produces a roughly constant
deceleration until the speed approaches zero, then tapers off. Throttle
remains at zero throughout the maneuver, indicating that the controller
correctly switches from drive to brake domain.

The resulting stopping distance and deceleration rate depend on the chosen
gains and the assumed brake torque capability, but are consistent with the
longitudinal model from Chapter~\ref{modeling}. The simulated response is
free of chatter at low speeds.
% Figure: Simulated braking response (speed vs time, brake command)
\begin{figure}[h]
    \centering
    % TODO replace with actual Simulink screenshot
    \includegraphics[width=\linewidth]{figures/Results/Figure_2.png}
    \caption{Results for Step Response to brake Controller}
    \label{fig:drive_response}
\end{figure}
\clearpage
\subsection{Steering Angle Controller}
\label{sec:sim_steer_results}

The steering controller was tested in simulation using a step in desired
steering angle. This approximates the behavior needed for both lane changes
and curvature following. The test procedure is:

\begin{enumerate}
    \item Initialize the vehicle with steering angle $\delta = 0$,
    \item At $t = 0$, command a step in steering angle $\delta^*$ within the
          physical limits of the mechanism,
    \item Record actual steering angle, steering command, and encoder counts.
\end{enumerate}

The simulated steering response follows the commanded angle with a modest rise
time and minimal overshoot. The response is dominated by the mechanical
bandwidth of the steering motor and the gains chosen for the steering PI
controller. Saturation and rate limits on the steering effort prevent
unrealistic accelerations at the steering rack.

The encoder feedback tracks the commanded position without significant steady
error once the controller has settled. In simulation, the quantization from
encoder counts is visible but does not cause instability.

% Figure: Simulated steering step response (steering angle vs time)
\begin{figure}[h]
    \centering
    % TODO replace with actual Simulink screenshot
    \includegraphics[width=\linewidth]{figures/Results/Figure_3.png}
    \caption{Results for Step Response to brake Controller}
    \label{fig:drive_response}
\end{figure}
\clearpage

\section{Hardware-in-the-Loop and Bench Testing}
\label{sec:hil_results}

Beyond pure Gazebo runs, the controller firmware was exercised in a
hardware-in-the-loop configuration. In this mode, the C2000 board runs the
Simulink-generated code and exchanges commands and encoder counts either with
Gazebo (through the simulation bridge) or with the real actuators on the cart.

\subsection{Test Setup}
\label{sec:hil_setup}

For bench testing, the C2000 was connected to:

\begin{itemize}
    \item The Kelly motor controller for traction drive,
    \item The steering motor driver and encoder,
    \item The brake actuator driver and feedback,
    \item A USB--serial link to a development machine running ROS~2.
\end{itemize}

Initial tests were performed with the rear wheels off the ground to avoid
uncontrolled vehicle motion. This allowed the speed and brake controllers to
be exercised without loading the drivetrain, and the steering controller to be
tested through its full range.

% Figure: Photo or diagram of HIL / bench test setup

\subsection{Longitudinal Control Results on Hardware}
\label{sec:hil_long_results}

On the bench, the speed controller was driven by synthetic step commands sent
from a ROS~2 node over the serial interface. The motor speed was measured from
the encoder and converted to linear speed using the wheel radius and gear
ratio.

Compared to simulation, the measured response shows similar shape but with a
slightly longer rise time, reflecting the real inertia and friction of the
drivetrain. The controller still converges to the commanded speed without
sustained oscillation, and the steady-state error remains small once the
system has settled.

The brake controller was tested by commanding transitions from a finite speed
down to zero. With the cart on stands, the brake actuator consistently brought
the wheels to a stop. Some differences from simulation appear in the last
portion of the response due to actuator backlash and mechanical compliance,
but the overall behavior matches the intended design.

% Figure: Measured speed step response on hardware
% Figure: Measured braking response on hardware

\subsection{Steering Control Results on Hardware}
\label{sec:hil_steer_results}

The steering controller was validated by commanding steps in desired steering
angle while logging encoder counts and motor current. With the front wheels
off the ground, the controller tracks setpoints across the usable range of
steering angles.

The measured response shows that the controller reaches the commanded angle
without excessive overshoot. Some small steady-state offsets appear at the
extremes of travel, which are attributed to mechanical limits and encoder
indexing. In practice, these offsets are within a few encoder counts and are
acceptable for low-speed maneuvering.

Under load on the ground, the steering motor current increases as expected,
but the closed-loop response remains stable. Further tuning could reduce the
time to reach large steering commands under load, but the basic control
structure is sound.

% Figure: Measured steering step response on hardware

\section{Summary and Limitations}
\label{sec:results_summary}

Overall, the results demonstrate that:

\begin{itemize}
    \item The longitudinal speed and brake controllers behave as intended in
          simulation, with smooth responses and small steady-state error,
    \item The steering controller tracks commanded angles within the mechanical
          limits of the steering mechanism,
    \item The C2000 firmware, serial interface, and ROS~2 bridge operate
          correctly in both simulation and bench tests.
\end{itemize}

The main limitations of the current results are:

\begin{itemize}
    \item Full GPS waypoint following, including closed-loop path tracking with
          the high level controller, was not completed in time for this report.
          As a result, there are no end-to-end autonomous drive runs to
          present.
    \item On-cart testing of the controllers was constrained by available time
          and test conditions, so the data set is limited compared to what
          would be required for full production validation.
\end{itemize}

Despite these limitations, the combination of simulation-based validation and
hardware tests confirms that the low-level control architecture is viable and
that the software and hardware layers integrate as designed. The step response
figures and HIL plots to be added in this chapter will serve as a concise
visual summary of the project’s technical outcomes.
