%Shift around citations. Maybe cite Nise and Franklin? Double check where I'm citing Iannou. Most of this is fairly bogstandard controls math though.
%Draw up some system block diagrams to pad stuff out. That'd be dope I think
\chapter{Low-Level Control Design}
\label{chap:low_level_control}

This chapter presents the low-level controllers used to regulate the golf cart's
longitudinal speed and steering angle. The combined planar model in
Chapter~\ref{modeling} provides the relationship between motor torque, brake
torque, steering angle, and platform motion. That model is used here as the basis
for control design.

This control layer maps a speed setpoint $v^*(t)$ and a steering setpoitn
$\delta^*(t)$ to actuator commands for the traction motor, friction brake, and
steering motor. Three controllers are used:

\begin{itemize}
  \item a drive motor controller that regulates longitudinal speed using motor
        torque,
  \item a brake controller that regulates longitudinal speed using friction
        brake torque,
  \item a steering controller that regulates the front wheel steering angle.
\end{itemize}

\section{Control Objectives and Architecture}
\label{sec:control_objectives}

The low-level control layer has two main objectives:

\begin{enumerate}
  \item Track a commanded longitudinal speed $v^*(t)$ in the operating region
        $0 \le v \le 20~\mathrm{mph}$ using the traction motor and friction brake.
  \item Track a commanded steering angle $\delta^*(t)$ of the front axle,
        consistent with the kinematic bicycle model in
        Section~\ref{lateral_kinematics}.
\end{enumerate}

A high-level controller generates the setpoints $v^*(t)$ and
$\delta^*(t)$ based on its internal error minimization calculations. The low-level controllers operate on
the tracking errors
\begin{equation}
    e_v(t) = v^*(t) - v(t),
    \qquad
    e_\delta(t) = \delta^*(t) - \delta(t),
    \label{eq:ctrl_errors}
\end{equation}
and produce commands for the motor controller, brake actuator, and steering
drive.

The longitudinal loop behaves similar to a cruise–control style system, with a speed
regulator acting on throttle and brake and a higher-level module providing
$v^*(t)$. The steering controller acts as a position servo for $\delta$.

The following control-specific assumptions are made:

\begin{itemize}
  \item The Kelly KLS motor controller and traction motor are modeled as an
        approximately linear gain between a normalized motor command and an
        effective drive torque. The brake actuator is modeled in the same way
        for brake torque.
  \item The design is restricted to forward motion; behavior very close to
        zero speed is handled by a stopped/hold mode in implementation.
\end{itemize}

\section{Longitudinal Speed Controller Design}
\label{sec:longitudinal_control}

The longitudinal dynamics in Section~\ref{longitudinal_dynamics} and
Section~\ref{combined_model} can be written as
\begin{equation}
    \dot v =
    \frac{\eta_d i_{\text{tot}}}{m_\text{eq} r_w}\,T_\text{motor}
    - \frac{1}{m_\text{eq} r_w}\,T_\text{brake}
    - \frac{C_{rr} m g}{m_\text{eq}},
    \label{eq:long_ctrl_full}
\end{equation}
where $T_\text{motor}$ and $T_\text{brake}$ are the effective wheel torques.
The rolling resistance term is treated as a constant disturbance for a given
operating condition.

The dynamic model can be simplified based on the domain the error signal exists in.
Let $u_\text{th} \in [0,1]$ and $u_\text{br} \in [0,1]$
be normalized throttle and brake commands. In the drive domain (brake
released) and the brake domain (throttle released), the dynamics are modeled as
\begin{align}
    \dot v &= g_d\,u_\text{th} - d,
    && \text{(drive domain)},
    \label{eq:drive_domain_plant} \\[0.25em]
    \dot v &= - g_b\,u_\text{br} - d,
    && \text{(brake domain)},
    \label{eq:brake_domain_plant}
\end{align}
where $g_d > 0$ and $g_b > 0$ are effectively gains and $d$ collects the rolling
resistance contribution. The corresponding transfer functions from command to
speed (with $d$ neglected in the nominal model) are
\begin{equation}
    G_d(s) = \frac{V(s)}{U_\text{th}(s)} = \frac{g_d}{s},
    \qquad
    G_b(s) = \frac{V(s)}{U_\text{br}(s)} = -\,\frac{g_b}{s}.
    \label{eq:long_plant_tf}
\end{equation}

Domain separation between drive and brake is implemented by a simple deadband
on the speed error. Let $e_v = v^* - v$. Then
\begin{equation}
    T_\text{motor}(t) =
    \begin{cases}
        T_\text{drive}(t), & e_v(t) > \Delta_v, \\
        0,                 & e_v(t) \le \Delta_v,
    \end{cases}
    \qquad
    T_\text{brake}(t) =
    \begin{cases}
        T_\text{brake,cmd}(t), & e_v(t) < -\Delta_v, \\
        0,                      & e_v(t) \ge -\Delta_v,
    \end{cases}
    \label{eq:domain_switching_ctrl}
\end{equation}
where $\Delta_v$ is a small deadband threshold, and $T_\text{drive}$ and
$T_\text{brake,cmd}$ are proportional to $u_\text{th}$ and $u_\text{br}$ through
the actuator gains.

\subsection{Drive Motor PI Controller}
\label{subsec:drive_controller}

In the drive domain, the plant is the integrator
\begin{equation}
    G_d(s) = \frac{g_d}{s}.
\end{equation}
The drive controller is a PI compensator acting on the error $e_v = v^* - v$
with transfer function
\begin{equation}
    C_v(s) = K_{p,v} + \frac{K_{i,v}}{s}.
    \label{eq:drive_pi_tf}
\end{equation}

With unity feedback, the closed-loop transfer function from $v^*$ to $v$ is
\begin{equation}
    \frac{V(s)}{V^*(s)} =
    \frac{C_v(s) G_d(s)}{1 + C_v(s) G_d(s)} =
    \frac{g_d\bigl(K_{p,v} s + K_{i,v}\bigr)}
         {s^2 + g_d K_{p,v} s + g_d K_{i,v}}.
    \label{eq:drive_cl_tf}
\end{equation}
The denominator in \eqref{eq:drive_cl_tf} defines the closed-loop
characteristic polynomial
\begin{equation}
    s^2 + g_d K_{p,v}\,s + g_d K_{i,v} = 0.
    \label{eq:drive_char_poly}
\end{equation}

To obtain a desired second-order response, \eqref{eq:drive_char_poly} is
matched to
\begin{equation}
    s^2 + 2 \zeta_v \omega_{n,v} s + \omega_{n,v}^2 = 0,
    \label{eq:second_order_form}
\end{equation}
where $\zeta_v$ is the damping ratio and $\omega_{n,v}$ is the natural
frequency of the speed loop. Equating coefficients gives
\begin{equation}
    K_{p,v} = \frac{2 \zeta_v \omega_{n,v}}{g_d},
    \qquad
    K_{i,v} = \frac{\omega_{n,v}^2}{g_d}.
    \label{eq:drive_gains}
\end{equation}

Integral action in $C_v(s)$ cancels the steady-state effect of the constant
disturbance $d$ in \eqref{eq:drive_domain_plant} for step changes in the speed
setpoint.

\subsection{Brake PI Controller}
\label{subsec:brake_controller}

In the brake domain, the plant is
\begin{equation}
    G_b(s) = -\,\frac{g_b}{s}.
\end{equation}
The brake loop uses a PI controller acting on the error
\begin{equation}
    e_b(t) = v(t) - v^*(t),
    \label{eq:brake_error}
\end{equation}
which is positive when the vehicle is faster than the setpoint. The controller
transfer function is
\begin{equation}
    C_b(s) = K_{p,b} + \frac{K_{i,b}}{s}.
    \label{eq:brake_pi_tf}
\end{equation}

With unity feedback and the sign convention in \eqref{eq:brake_error}, the
closed-loop characteristic polynomial again takes the form
\begin{equation}
    s^2 + g_b K_{p,b}\,s + g_b K_{i,b} = 0.
\end{equation}
Matching to \eqref{eq:second_order_form} yields
\begin{equation}
    K_{p,b} = \frac{2 \zeta_b \omega_{n,b}}{g_b},
    \qquad
    K_{i,b} = \frac{\omega_{n,b}^2}{g_b},
    \label{eq:brake_gains}
\end{equation}
with $(\zeta_b,\omega_{n,b})$ chosen separately from the drive loop. In
practice, braking is tuned to be more heavily damped (larger $\zeta_b$) and
sometimes slower (smaller $\omega_{n,b}$) than throttle actuation to improve
ride comfort and limit jerk \cite{rajamani,ioannou_xu}.

\section{Steering Controller Design}
\label{sec:steering_control}

The steering controller regulates the front wheel steering angle $\delta$ to
track a setpoint $\delta^*(t)$ generated by the high-level lateral controller.
At the hardware level, the steering column is driven by a DC motor through a
motor controller.

The steering mechanism is modeled as a first-order integrator,
\begin{equation}
    \dot \delta(t) = k_s\,u_\delta(t),
    \label{eq:steering_plant}
\end{equation}
where $u_\delta$ is a normalized steering command and $k_s > 0$ is an effective
gain. The corresponding steering plant transfer function is
\begin{equation}
    G_s(s) = \frac{\Delta(s)}{U_\delta(s)} = \frac{k_s}{s},
    \label{eq:steering_plant_tf}
\end{equation}
where $\Delta(s)$ and $U_\delta(s)$ are the Laplace transforms of $\delta(t)$
and $u_\delta(t)$, respectively.

\subsection{Steering PI Controller}
\label{subsec:steering_pi}

The steering error is defined as
\begin{equation}
    e_\delta(t) = \delta^*(t) - \delta(t),
    \label{eq:steering_error}
\end{equation}
and the controller uses a PI structure with transfer function
\begin{equation}
    C_\delta(s) = K_{p,\delta} + \frac{K_{i,\delta}}{s}.
    \label{eq:steering_pi_tf}
\end{equation}

With unity feedback and plant \eqref{eq:steering_plant_tf}, the closed-loop
transfer function from $\delta^*$ to $\delta$ is
\begin{equation}
    \frac{\Delta(s)}{\Delta^*(s)} =
    \frac{C_\delta(s) G_s(s)}{1 + C_\delta(s) G_s(s)} =
    \frac{k_s\bigl(K_{p,\delta} s + K_{i,\delta}\bigr)}
         {s^2 + k_s K_{p,\delta} s + k_s K_{i,\delta}}.
    \label{eq:steering_cl_tf}
\end{equation}
The denominator defines the closed-loop characteristic polynomial
\begin{equation}
    s^2 + k_s K_{p,\delta}\,s + k_s K_{i,\delta} = 0.
\end{equation}

Matching to the standard second-order form \eqref{eq:second_order_form} gives
\begin{equation}
    K_{p,\delta} = \frac{2 \zeta_\delta \omega_{n,\delta}}{k_s},
    \qquad
    K_{i,\delta} = \frac{\omega_{n,\delta}^2}{k_s},
    \label{eq:steering_gains}
\end{equation}
where $(\zeta_\delta,\omega_{n,\delta})$ are the damping ratio and natural
frequency of the steering loop.

\section{Summary}
\label{sec:control_summary}

This chapter presented the low-level control design for the golf cart platform using the combined planar model in
Chapter~\ref{modeling} as a basis. The equations outlined above describe the systems by which the platform 
translates its desired high level setpoints into motion of the platform by way of the low level control systems, constructed 
as standard PI controllers, map measured error to actuator input.