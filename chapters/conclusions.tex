\chapter{Conclusion}
\label{chap:conclusion}

This chapter summarizes the work completed in this project and outlines
logical next steps for the autonomous golf cart platform. The focus is on
what was built, what was validated, and what remains.

\section{Summary of Contributions}
\label{sec:contrib_summary}

This project extends the University of West Florida autonomous golf cart
platform with a complete low level control stack, a physics-based simulation
environment, and a defined high level control architecture.

On the modeling side, Chapter~\ref{modeling} derived longitudinal and lateral
vehicle models suitable for control design. The drivetrain model captures the
relationship between motor torque, gear ratio, and wheel force, while the
planar bicycle model provides a basis for steering control and path tracking.

Using these models, Chapter~\ref{chap:low_level_control_design} designed PI
controllers for traction, braking, and steering. The controller structure is
simple enough to implement on an embedded microcontroller but still grounded
in the underlying physics of the vehicle.

Chapter~\ref{chap:simulation} introduced a Gazebo-based simulation of the
golf cart, including a URDF model, an Ackermann steering plugin, and a ROS~2
bridge that connects the simulator to the same command interface used by the
physical cart. This environment allows the controller firmware to be tested
against a dynamic vehicle model before being deployed to hardware.

On the embedded side, the project implemented the C2000 low level controller
in Simulink, including the serial command interface, encoder decoding, and
actuator outputs. The same model is used for both hardware-in-the-loop tests
and on-cart operation.

Finally, Chapter~\ref{chap:high_level_control} defined the high level control
architecture on the Jetson. This includes ROS~2 nodes for GPS and compass
sensors, a serial interface to the C2000, and a planned waypoint-following
controller. While the full GPS navigation behavior was not completed, the
interfaces and message flows are in place.

\section{Project Outcomes}
\label{sec:project_outcomes}

The main technical outcomes are:

\begin{itemize}
    \item A consistent set of vehicle models that tie together the physical
          parameters of the cart, the controller design, and the simulation
          environment,
    \item A working low level controller on the C2000 that can command drive,
          brake, and steering actuators using a unified serial interface,
    \item A Gazebo simulation setup that exercises the same firmware and
          command protocol as the real hardware,
    \item A ROS~2-based high level software stack on the Jetson with tested
          sensor drivers and a defined command path to the low level
          controller.
\end{itemize}

Simulation and bench tests in Chapter~\ref{chap:results} show that the low
level controllers behave as expected. Speed and brake responses track
setpoints smoothly, and the steering controller follows commanded angles
within the mechanical limits of the steering system. The serial interface and
ROS~2 bridge operate as designed in both simulation and hardware tests.

The main limitation is that full end-to-end autonomous driving with GPS
waypoints was not realized within the project timeline. The system currently
stops at the point where high level commands can be generated and sent to the
C2000, but closed-loop navigation using those commands on the real vehicle
remains future work.

\section{Closing Remarks}
\label{sec:closing_remarks}

The work presented in this report takes the autonomous golf cart platform
from a mostly mechanical project to an integrated system with a defined
control architecture, working low level controllers, and a realistic
simulation environment. While full autonomous navigation is not yet
demonstrated, the pieces needed to reach that goal are now in place and
coherent.

Future students can build on this foundation rather than starting from
scratch. Completing the GPS waypoint follower, adding state estimation and
safety layers, and expanding perception will move the platform closer to a
practical, campus-scale autonomous vehicle.
