\chapter{Vehicle Hardware Rework}
[Go from initial condition of the vehicle to current state. Rewire etc.]
This section details the work done on the platform to convert it from the original control
systems to the updated control systems. This work encompasses component selection, electrical rework,
and  installation of new components. The original state of the vehicle is described, and a narrative 
of the process is provided.

\section{Initial Vehicle Condition}
[Describe as-is vehicle state at project start]
The initial vehicle condition had all power and signal wiring for the platform routed through plastic cable harnessing.
Wires were routed end to end and spliced mid-harness to make connections between platform level components, low-level components,
and high-level components. Evaluating this design, it was found to have accrued damage over time in the form of wire shorts and frays.

%What else should go here?
\section{Component Update}
[Discuss updated components (arduino/rpi to c2000/jetson)]
Originally, the low level control system was comprised of several arduino 8-bit microcontrollers slaved to an ESP32 32-bit microcontroller.
This system was evaluated, and determined to be entirely upgradeable to a single C2000 based microcontroller. The interfaces that the new microcontroller
overtakes are as follows:
- Drive Motor Controller interface
- Brake Actuator interface
- Throttle Pedal interface
- Steering Motor Controller interface
- Steering Shaft Encoder interface
- High-Level to Low-Level bridge interface

Additionally, the high-level system originally comprised of a raspberry pi compute module connected to an Apple iMac Desktop. This 
has been replaced with a single NVIDIA Jetson Orin Nano compute module. This compute module handles the following responsibilities:
- High-Level sensor monitoring such as GPS and LiDAR
- Complex algorithm calculation such as path planning and obstacle detection/avoidance

\section{Electrical Rework}
[Discuss golfcart rewire from wiring harness to DIN rails]
The physical platform was ultimately rewired. The wiring harness was converted to DIN rails. This is effectively a conversion from point to point 
wiring to a bus sytle system. The advantages to this are that it enables alterations and extensions of the current design to be modular and straightforward.
