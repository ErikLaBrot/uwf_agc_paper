\chapter{Vehicle Hardware Rework}
\label{chap:rework}

This chapter describes the changes made to the physical platform to support the
control and software design in later chapters. The focus is on the control
hardware update and the electrical rework that provides a more robust interface
between the vehicle, the C2000 microcontroller, and the Jetson.

\section{Initial Vehicle Condition}
\label{sec:rework_initial}

When this stage of the project began, the cart had already been converted from
its original gasoline drivetrain to an electric traction system. The vehicle
carried a traction battery pack, a motor controller, steering and brake
actuators, and an assortment of low level control boards.

Power and signal wiring were routed along the frame in plastic cable harnesses.
Most connections were made point-to-point. New devices were added by splicing
into existing runs and extending the harness. The result worked, but it was
not easy to trace individual circuits or change control hardware without
disturbing other parts of the system.

For the control work in this report, it was more effective to replace this
wiring approach than to continue to extend it. The goal of the rework was to
create a layout that is easier to understand, document, and maintain.

\section{Control Hardware Update}
\label{sec:rework_control_hw}

The original low level control used multiple microcontrollers. Several
Arduino-class 8-bit boards handled local actuator and sensor tasks, and an
ESP32 acted as a coordinator between them and the higher level computer. This
distributed arrangement increased the number of devices that had to be
configured and maintained.

In the updated design, these boards are replaced by a single TI F28379D
C2000-based controller. The C2000 is responsible for:

\begin{itemize}
    \item generating throttle and brake commands for the traction inverter,
    \item commanding the steering actuator and reading the steering encoder,
    \item reading speed feedback from the drive motor encoder,
    \item exchanging speed and steering setpoints with the Jetson over serial.
\end{itemize}

Using one controller for all low level tasks simplifies both hardware and
software. All actuator interfaces, feedback signals, and control logic reside
on a single board. This matches the control structure used in the modeling and
low level controller design chapters.

On the high level side, an NVIDIA Jetson Orin Nano serves as the primary
compute unit for autonomous navigation. The Jetson connects to the C2000
through a USB–serial link and to sensors such as GPS, compass, and LiDAR
through their respective interfaces. The rework ensures that these connections
terminate on clear, labeled points instead of ad hoc splices.

\section{Electrical Rework}
\label{sec:rework_electrical}

The electrical rework centers on moving from embedded harness splices to a
panel-based layout using DIN rail and terminal blocks. Power distribution,
signal routing, and controller connections are brought into a single point
mounted on the vehicle.

The main elements of the new layout are:

\begin{itemize}
    \item a set of DIN-rail mounted terminal blocks for traction battery,
          low voltage supply, and signal wiring,
    \item dedicated terminals for each actuator and sensor connection
          (traction inverter, steering motor, brake actuator, encoders,
          GPS, compass, LiDAR),
    \item separation of high-current power runs from low-voltage control and
          sensor wiring,
    \item labeled jumpers and fusing for the 12~V and 5~V control supplies.
\end{itemize}

The C2000 and Jetson are wired into this panel rather than directly into the
vehicle harness. Each I/O line from the controllers appears on a labeled
terminal, then routes outward to the corresponding device. This provides a
single place to probe signals, insert temporary instrumentation, or reroute
connections during testing.

The traction battery feeds a DC–DC converter that supplies the low voltage
control bus. From there, regulated rails are distributed through the terminal
blocks to the motor controller, actuators, and sensors. This arrangement keeps
the power path explicit and makes it clear which devices share a supply.

\section{Resulting Platform}
\label{sec:rework_result}

The hardware rework produces a vehicle that is easier to work with at the
control level. The main improvements are:

\begin{itemize}
    \item low level control consolidated on the C2000,
    \item a clear electrical interface between the vehicle, the C2000, and the
          Jetson,
    \item panel-based power and signal distribution with labeled terminals,
    \item reduced dependence on in-harness splices and undocumented wiring.
\end{itemize}

This design supports the modeling, controller design, and implementation
work in the following chapters. It also provides future work a defined set of
connection points for adding sensors, actuators, or safety systems without
repeating another full wiring rework.
